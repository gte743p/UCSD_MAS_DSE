\documentclass{article}

\usepackage{fancyhdr}
\usepackage{extramarks}
\usepackage{amsmath}
\usepackage{amsthm}
\usepackage{amsfonts}
\usepackage{tikz}
\usepackage[plain]{algorithm}
\usepackage{algpseudocode}
\usepackage{enumitem}
\usepackage{mathtools}
\usepackage{amssymb}
\usetikzlibrary{automata,positioning}
\usepackage{pgfplots}

%
% Basic Document Settings
%

\topmargin=-0.45in
\evensidemargin=0in
\oddsidemargin=0in
\textwidth=6.5in
\textheight=9.0in
\headsep=0.25in

\linespread{1.1}

\pagestyle{fancy}
\lhead{\hmwkClass\ \hmwkTitle}
\rhead{\hmwkAuthorName}
\lfoot{\lastxmark}
\cfoot{\thepage}

\renewcommand\headrulewidth{0.4pt}
\renewcommand\footrulewidth{0.4pt}

\setlength\parindent{0pt}

%
% Added stuff
%

\DeclarePairedDelimiter\abs{\lvert}{\rvert}
\setenumerate[0]{label={(\alph*)}}

\pgfplotsset{my style/.append style={axis x line=middle, axis y line=
           middle, xlabel={$x$}, ylabel={$y$}, axis equal }}

%
% Create Problem Sections
%

\newcommand{\enterProblemHeader}[1]{
    \nobreak\extramarks{}{Problem \arabic{#1} continued on next page\ldots}\nobreak{}
    \nobreak\extramarks{Problem \arabic{#1} (continued)}{Problem \arabic{#1} continued on next page\ldots}\nobreak{}
}

\newcommand{\exitProblemHeader}[1]{
    \nobreak\extramarks{Problem \arabic{#1} (continued)}{Problem \arabic{#1} continued on next page\ldots}\nobreak{}
    \stepcounter{#1}
    \nobreak\extramarks{Problem \arabic{#1}}{}\nobreak{}
}

\setcounter{secnumdepth}{0}
\newcounter{partCounter}
\newcounter{homeworkProblemCounter}
\setcounter{homeworkProblemCounter}{1}
\nobreak\extramarks{Problem \arabic{homeworkProblemCounter}}{}\nobreak{}

%
% Homework Problem Environment
%
% This environment takes an optional argument. When given, it will adjust the
% problem counter. This is useful for when the problems given for your
% assignment aren't sequential. See the last 3 problems of this template for an
% example.
%
\newenvironment{homeworkProblem}[1][-1]{
    \ifnum#1>0
        \setcounter{homeworkProblemCounter}{#1}
    \fi
    \section{Problem \arabic{homeworkProblemCounter}}
    \setcounter{partCounter}{1}
    \enterProblemHeader{homeworkProblemCounter}
}{
    \exitProblemHeader{homeworkProblemCounter}
}

%
% Homework Details
%   - Title
%   - Due date
%   - Class
%   - Section/Time
%   - Instructor
%   - Author
%

\newcommand{\hmwkTitle}{Worksheet\ \#6 - Generative Models 2}
\newcommand{\hmwkDueDate}{February 17, 2017}
\newcommand{\hmwkClass}{DSE 210}
\newcommand{\hmwkClassTime}{}
\newcommand{\hmwkClassInstructor}{Professor: A. Enis \c{C}etin}
\newcommand{\hmwkClassTA}{Teaching Assistant: Shivani Agrawal}
\newcommand{\hmwkAuthorName}{\textbf{Joshua Wilson} \and \textbf{A53228518}}

%
% Title Page
%

\title{
    \vspace{2in}
    \textmd{\textbf{\hmwkClass:\ \hmwkTitle}}\\
    \vspace{0.1in}\large{\textit{\hmwkClassInstructor}}\\
    \vspace{0.1in}\large{\textit{\hmwkClassTA}}
    \vspace{3in}
}

\author{\hmwkAuthorName}
\date{}

\renewcommand{\part}[1]{\textbf{\large Part \Alph{partCounter}}\stepcounter{partCounter}\\}

%
% Various Helper Commands
%

% Useful for algorithms
\newcommand{\alg}[1]{\textsc{\bfseries \footnotesize #1}}

% For derivatives
\newcommand{\deriv}[1]{\frac{\mathrm{d}}{\mathrm{d}x} (#1)}

% For partial derivatives
\newcommand{\pderiv}[2]{\frac{\partial}{\partial #1} (#2)}

% Integral dx
\newcommand{\dx}{\mathrm{d}x}

% Alias for the Solution section header
\newcommand{\solution}{\textbf{\large Solution}}

% Probability commands
\newcommand{\E}{\mathrm{E}}
\newcommand{\Var}{\mathrm{Var}}
\newcommand{\Cov}{\mathrm{Cov}}
\newcommand{\Bias}{\mathrm{Bias}}
\newcommand*{\Perm}[2]{{}_{#1}\!P_{#2}}%
\newcommand*{\Comb}[2]{{}_{#1}C_{#2}}%
\newcommand\given[1][]{\:#1\vert\:}
\DeclareMathOperator*{\argmax}{arg\,max}

\begin{document}

\maketitle

\pagebreak

\begin{homeworkProblem}
Given:
\begin{equation*}
\begin{aligned}[c]
\pi(\text{happy}) = \textstyle\frac{3}{4}\\
p(\text{talks a lot} \given \text{happy}) = \textstyle\frac{2}{3}\\
p(\text{talks a little}\given \text{happy}) = \textstyle\frac{1}{6}\\
p(\text{silent} \given \text{happy}) = \textstyle\frac{1}{6}
\end{aligned}
\begin{aligned}[c]
 \qquad \qquad \qquad\\
 \qquad \qquad \qquad\\
 \qquad \qquad \qquad\\
 \qquad \qquad \qquad
\end{aligned}
\begin{aligned}[c]
\pi(\text{sad}) = \textstyle\frac{1}{4}\\
p(\text{talks a lot} \given \text{sad}) = \textstyle\frac{1}{6}\\
p(\text{talks a little}\given \text{sad}) = \textstyle\frac{1}{6}\\
p(\text{silent} \given \text{sad}) = \textstyle\frac{2}{3}
\end{aligned}
\end{equation*}

\begin{enumerate}
\item
To calculate the most likely mood, we determine the probabilities of each mood given the fact that he is only talking a little: \\ \\
$\begin{aligned}
p(\text{happy} \given \text{talks a little}) & = \cfrac{p(\text{talks a little} \given \text{happy}) \times p(\text{happy})}{p(\text{talks a little})} \\
	& = \cfrac{p(\text{talks a little} \given \text{happy}) \times p(\text{happy})}{p(\text{talks a little} \given \text{happy}) \times p(\text{happy}) 
	+ p(\text{talks a little} \given \text{sad}) \times p(\text{sad})} \\
	& = \cfrac{\frac{1}{6} \times \frac{3}{4}}{\frac{1}{6} \times \frac{3}{4} + \frac{1}{6} \times \frac{1}{4}} 
	= \cfrac{\frac{3}{24}}{\frac{3}{24} + \frac{1}{24}}
	= \cfrac{\frac{3}{24}}{\frac{4}{24}}
	= \cfrac{3}{4} \\ \\
p(\text{sad} \given \text{talks a little}) & = \cfrac{p(\text{talks a little} \given \text{sad}) \times p(\text{sad})}{p(\text{talks a little})} \\
	& = \cfrac{p(\text{talks a little} \given \text{sad}) \times p(\text{sad})}{p(\text{talks a little} \given \text{sad}) \times p(\text{sad}) 
	+ p(\text{talks a little} \given \text{sad}) \times p(\text{sad})} \\
	& = \cfrac{\frac{1}{6} \times \frac{1}{4}}{\frac{1}{6} \times \frac{1}{4} + \frac{1}{6} \times \frac{3}{4}} 
	= \cfrac{\frac{1}{24}}{\frac{1}{24} + \frac{3}{24}}
	= \cfrac{\frac{1}{24}}{\frac{4}{24}}
	= \cfrac{1}{4}
\end{aligned}$ \\ \\ 
Since $p(\text{happy} \given \text{talks a little}) > p(\text{sad} \given \text{talks a little})$, the most likely mood is $\boxed{\text{happy}}$ \\

\item
The probability of the prediction in 1(a) being incorrect is equal to $p(\text{sad} \given \text{talks a little})$, which was calculated above and is equal to $\boxed{\cfrac{1}{4}}$
\end{enumerate}
\end{homeworkProblem}

\begin{homeworkProblem}
To find the optimal classifier $h^{\ast}$, we need to find $h^{\ast}(x) = \argmax _j \pi _j P_j(x)$. \\
\begin{equation*}
\begin{aligned}
\pi _1 P_1(x) & = \begin{cases}
    \frac{1}{3} \times \frac{7}{8} = \frac{7}{24}, & \text{if } -1 \leq x < 0 \\
    \frac{1}{3} \times \frac{1}{8} = \frac{1}{24}, & \text{if } \ \ \ \ 0 \leq x \leq 1
  \end{cases}, \\
\pi _2 P_2(x) & = \begin{cases}
    \frac{1}{6} \times 0 = 0, & \text{if } -1 \leq x < 0 \\
    \frac{1}{6} \times 1 = \frac{1}{6}, & \text{if } \ \ \ \ 0 \leq x \leq 1
  \end{cases}, \\
\pi _3 P_3(x) & = \begin{cases}
    \frac{1}{2} \times \frac{1}{2} = \frac{1}{4}, & \text{if } -1 \leq x < 0 \\
    \frac{1}{2} \times \frac{1}{2} = \frac{1}{4}, & \text{if } \ \ \ \ 0 \leq x \leq 1
  \end{cases}, \\
max\ \pi _j P_j(x) & = \begin{cases}
    \pi _1 P_1(x) = \frac{1}{3} \times \frac{7}{8} = \frac{7}{24}, & \text{if } -1 \leq x < 0 \\
    \pi _3 P_3(x) = \frac{1}{2} \times \frac{1}{2} = \frac{1}{4}, & \text{if } \ \ \ \ 0 \leq x \leq 1
  \end{cases}, \\
\therefore h^{\ast}(x) & = \begin{cases}
    1, & \text{if } -1 \leq x < 0 \\
    3, & \text{if } \ \ \ \ 0 \leq x \leq 1
    \end{cases}
\end{aligned}
\end{equation*}

\end{homeworkProblem}

\begin{homeworkProblem}
\begin{enumerate}
\item
positively correlated (although this probably varies by make, model, and type of vehicle; for example, some sports cars engineered to be very light would be expensive)
\item
positively correlated
\item
negatively correlated
\end{enumerate}
\end{homeworkProblem}

\begin{homeworkProblem}
The elements are perfectly correlated, i.e. correlation = $\boxed{1}$.
\end{homeworkProblem}

\newpage

\begin{homeworkProblem}
\begin{enumerate}
\item
$\mu_x = 2, \qquad \sigma_x = 1, \\
\mu_y = 2, \qquad \sigma_y = 0.5, \\
corr(x, y) = -0.5, \\ \\
\text{Mean vector} \begin{bmatrix}\mu_x \\ \mu_y\end{bmatrix} = \begin{bmatrix} 2 \\ 2 \end{bmatrix}
\text{, and covariance matrix } \Sigma = \begin{bmatrix}\Sigma_{xx} & \Sigma_{xy} \\ \Sigma_{yx} & \Sigma_{yy} \end{bmatrix}$, \\ \\
where:
\begin{equation*}
\begin{aligned}
\Sigma_{xx} & = var(x) = \sigma_x^2 = 1^2 = 1 \\
\Sigma_{yy} & = var(y) = \sigma_y^2 = 0.5^2 = 0.25 \\
\Sigma_{xy} & = \Sigma_{yx} = cov(x, y) = corr(x, y) \times \sigma_x \times \sigma_y = -0.5 \times 1 \times 0.5 = -0.25
\end{aligned}
\end{equation*}
$\therefore \boxed{\mu = \begin{bmatrix} 2 \\ 2 \end{bmatrix} \text{ and } \Sigma = \begin{bmatrix} 1 & -0.25 \\ -0.25 & 0.25 \end{bmatrix}}$

\item
$\mu_x = 1, \qquad \sigma_x = 1, \\
\mu_y = 1, \qquad \sigma_y = 1, \\
corr(x, y) = 1, \\ \\
\therefore \boxed{\mu = \begin{bmatrix} 1 \\ 1 \end{bmatrix} \text{ and } \Sigma = \begin{bmatrix} 1 & 1 \\ 1 & 1 \end{bmatrix}}$
\end{enumerate}
\end{homeworkProblem}

\newpage

\begin{homeworkProblem}
\begin{enumerate}
\item
$\mu = \begin{bmatrix} 0 \\ 0 \end{bmatrix}, \Sigma = \begin{bmatrix} 9 & 0 \\  0 & 1 \end{bmatrix}$, \\

Gaussian contour sketch: \\

\begin{tikzpicture}
         \begin{axis}[my style, xtick={-9,-6,...,9}, ytick={-9,-6,...,9}, xmin=-10, xmax=10,  ymin=-10, ymax=10]
	\end{axis}
\end{tikzpicture} \\

\item
$\mu = \begin{bmatrix} 0 \\ 0 \end{bmatrix}, \Sigma = \begin{bmatrix} 1 & -0.75 \\  -0.75 & 1 \end{bmatrix}$, \\

Gaussian contour sketch: \\

\begin{tikzpicture}
    \begin{axis}[my style, xtick={-3,-2,...,3}, ytick={-3,-2,...,3}, xmin=-3, xmax=3,  ymin=-3, ymax=3]
    \end{axis}
\end{tikzpicture}

\end{enumerate}
\end{homeworkProblem}

\begin{homeworkProblem}
See Worksheet6\_Problem7.ipynb notebook at https://github.com/mas-dse/jsw037/tree/master/DSE210.
\end{homeworkProblem}

\newpage

\begin{homeworkProblem}
Given:  $\vec{w} = \begin{bmatrix} -3 \\ 4 \end{bmatrix}, \theta = 12\text{.  Find the decision boundary in } \mathbb{R}^2$.\\ \\ \\

The boundary will be orthogonal to $\vec{w}$, and the minimum distance between any point along the boundary and the origin will be $\cfrac{\theta}{\|\vec{w}\|}. \\ \\
\|\vec{w}\| = \sqrt{(-3)^2 + 4^2} = \sqrt{9 + 16} = \sqrt{25} = 5$, so $\cfrac{\theta}{\|\vec{w}\|} = \cfrac{12}{5}$.  \\ 

The unit vector in the direction of $\vec{w}$ is $\vec{u}_w = \cfrac{\vec{w}}{\|\vec{w}\|} = \begin{bmatrix} \frac{-3}{5} \\[6pt] \frac{4}{5} \end{bmatrix}$.  \\

The intersection of $\vec{w}$ and the boundary occurs at the point $\cfrac{\theta}{\|\vec{w}\|} \times \vec{u}_w = \cfrac{12}{5} \begin{bmatrix} \frac{-3}{5} \\[6pt] \frac{4}{5} \end{bmatrix} = \begin{bmatrix} \frac{-36}{25} \\[6pt] \frac{48}{25} \end{bmatrix}$. \\

If we define $\vec{v}$ to be a vector parallel to the boundary, we know that $\vec{w} \bullet \vec{v} = 0$, since the boundary is orthogonal to $\vec{w}$.  \\

$\vec{w} \bullet \vec{v} = \begin{bmatrix} -3 \\ 4 \end{bmatrix} \bullet \begin{bmatrix} v_1 \\ v_2 \end{bmatrix} = -3 v_1 + 4 v_2 = 0 \implies 4 v_2 = 3 v_1 \implies v_2 = \frac{3}{4} v_1$, so the slope of $\vec{v}$ is $\cfrac{3}{4}.$ \\

Now, since we know a point on the boundary and the slope of the boundary, we can find the $y = mx + b$ slope-intercept equation for the boundary line by using $x = \frac{-36}{25}, y = \frac{48}{25},$ and $m = \frac{3}{4}$:
\begin{equation*}
\cfrac{48}{25} = \cfrac{3}{4} \times \cfrac{-36}{25} + b \implies b = \cfrac{48}{25} - \cfrac{3}{4} \times \cfrac{-36}{25} = \cfrac{48}{25} + \cfrac{27}{25} = \cfrac{75}{25} = 3,
\end{equation*}
So the boundary line is defined by the equation $y = \cfrac{3}{4}\ x + 3$. \\ \\
Now we can set each variable to 0 to solve for where the boundary intersects each coordinate axis.  \\ \\
If $y = 0$, then $x = -4$, and if $x = 0$, then $y = 3$.  \\ \\
The boundary intersects the coordinate axes at points $\begin{bmatrix} 0 \\ 3 \end{bmatrix}$ and $\begin{bmatrix} -4 \\ 0 \end{bmatrix}$. \\ \\

\begin{tikzpicture}
         \begin{axis}[my style, minor tick num = 1]
		\addplot[domain=-15:5]{1/2 * x+5};
		\node (b) at (axis cs:2,4){boundary};
		\node (+) at (axis cs:	-2,2){+};
		\node (-) at (axis cs:-1.5,1.5){-};
		\node (source) at (axis cs:0,0){};
       		\node (destination) at (axis cs:-1,2){$\vec{w}$};
       		\draw[->](source)--(destination);
    \end{axis}
\end{tikzpicture}


\end{homeworkProblem}

\begin{homeworkProblem}
See Worksheet6\_Problem9.ipynb notebook at https://github.com/mas-dse/jsw037/tree/master/DSE210.
\end{homeworkProblem}


\end{document}




