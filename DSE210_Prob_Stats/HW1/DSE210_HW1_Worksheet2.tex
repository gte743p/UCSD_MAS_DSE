\documentclass{article}

\usepackage{fancyhdr}
\usepackage{extramarks}
\usepackage{amsmath}
\usepackage{amsthm}
\usepackage{amsfonts}
\usepackage{tikz}
\usepackage[plain]{algorithm}
\usepackage{algpseudocode}
\usepackage{enumitem}
\usepackage{mathtools}
\usepackage{amssymb}

\usetikzlibrary{automata,positioning}

%
% Basic Document Settings
%

\topmargin=-0.45in
\evensidemargin=0in
\oddsidemargin=0in
\textwidth=6.5in
\textheight=9.0in
\headsep=0.25in

\linespread{1.1}

\pagestyle{fancy}
\lhead{\hmwkClass\ \hmwkTitle}
\rhead{\hmwkAuthorName}
\lfoot{\lastxmark}
\cfoot{\thepage}

\renewcommand\headrulewidth{0.4pt}
\renewcommand\footrulewidth{0.4pt}

\setlength\parindent{0pt}

%
% Added stuff
%

\DeclarePairedDelimiter\abs{\lvert}{\rvert}
\setenumerate[0]{label={(\alph*)}}

%
% Create Problem Sections
%

\newcommand{\enterProblemHeader}[1]{
    \nobreak\extramarks{}{Problem \arabic{#1} continued on next page\ldots}\nobreak{}
    \nobreak\extramarks{Problem \arabic{#1} (continued)}{Problem \arabic{#1} continued on next page\ldots}\nobreak{}
}

\newcommand{\exitProblemHeader}[1]{
    \nobreak\extramarks{Problem \arabic{#1} (continued)}{Problem \arabic{#1} continued on next page\ldots}\nobreak{}
    \stepcounter{#1}
    \nobreak\extramarks{Problem \arabic{#1}}{}\nobreak{}
}

\setcounter{secnumdepth}{0}
\newcounter{partCounter}
\newcounter{homeworkProblemCounter}
\setcounter{homeworkProblemCounter}{1}
\nobreak\extramarks{Problem \arabic{homeworkProblemCounter}}{}\nobreak{}

%
% Homework Problem Environment
%
% This environment takes an optional argument. When given, it will adjust the
% problem counter. This is useful for when the problems given for your
% assignment aren't sequential. See the last 3 problems of this template for an
% example.
%
\newenvironment{homeworkProblem}[1][-1]{
    \ifnum#1>0
        \setcounter{homeworkProblemCounter}{#1}
    \fi
    \section{Problem \arabic{homeworkProblemCounter}}
    \setcounter{partCounter}{1}
    \enterProblemHeader{homeworkProblemCounter}
}{
    \exitProblemHeader{homeworkProblemCounter}
}

%
% Homework Details
%   - Title
%   - Due date
%   - Class
%   - Section/Time
%   - Instructor
%   - Author
%

\newcommand{\hmwkTitle}{Worksheet\ \#2 - Probability Spaces}
\newcommand{\hmwkDueDate}{January 20, 2017}
\newcommand{\hmwkClass}{DSE 210}
\newcommand{\hmwkClassTime}{}
\newcommand{\hmwkClassInstructor}{Professor: A. Enis \c{C}etin}
\newcommand{\hmwkClassTA}{Teaching Assistant: Shivani Agrawal}
\newcommand{\hmwkAuthorName}{\textbf{Joshua Wilson} \and \textbf{A53228518}}

%
% Title Page
%

\title{
    \vspace{2in}
    \textmd{\textbf{\hmwkClass:\ \hmwkTitle}}\\
    \vspace{0.1in}\large{\textit{\hmwkClassInstructor}}\\
    \vspace{0.1in}\large{\textit{\hmwkClassTA}}
    \vspace{3in}
}

\author{\hmwkAuthorName}
\date{}

\renewcommand{\part}[1]{\textbf{\large Part \Alph{partCounter}}\stepcounter{partCounter}\\}

%
% Various Helper Commands
%

% Useful for algorithms
\newcommand{\alg}[1]{\textsc{\bfseries \footnotesize #1}}

% For derivatives
\newcommand{\deriv}[1]{\frac{\mathrm{d}}{\mathrm{d}x} (#1)}

% For partial derivatives
\newcommand{\pderiv}[2]{\frac{\partial}{\partial #1} (#2)}

% Integral dx
\newcommand{\dx}{\mathrm{d}x}

% Alias for the Solution section header
\newcommand{\solution}{\textbf{\large Solution}}

\newcommand*{\Perm}[2]{{}_{#1}\!P_{#2}}%
\newcommand*{\Comb}[2]{{}_{#1}C_{#2}}%

% Probability commands: Expectation, Variance, Covariance, Bias
\newcommand{\E}{\mathrm{E}}
\newcommand{\Var}{\mathrm{Var}}
\newcommand{\Cov}{\mathrm{Cov}}
\newcommand{\Bias}{\mathrm{Bias}}

\begin{document}

\maketitle

\pagebreak

\begin{homeworkProblem}

\begin{enumerate}
\item $\Omega = \{A, B\}$
\item $\Omega = \{H, T\}$
\item $\Omega = \{months\} \times \{days\ of\ week\} = \{(Jan,\ Mon),\ (Jan,\ Tue),\  \ldots\ , (Dec,\ Sat),\ (Dec,\ Sun)\}$, \\ 
$\abs{\Omega} = 12 \times 7 = 84$
\item $\Omega = \{s1,\ s2,\ s3,\ \ldots,\ s10\}$
\item $\Omega = \{exterior\ colors\} \times \{interior\ colors\}$ = \{red, black, silver, green\} $\times$ \{black, beige\} $=$ \\ \{(red, black), (red, beige), (black, black), (black, beige), (silver, black), (silver, beige), (green, black), (green, beige)\}, \\
$\abs{\Omega} = 4 \times 2 = 8$

\end{enumerate}
\end{homeworkProblem}

\begin{homeworkProblem}[3]
\begin{enumerate}
\item $A \cap B \cap C$
\item $A \cup B \cup C$
\item $A \cap B \cap C^c$
\end{enumerate}
\end{homeworkProblem}

\begin{homeworkProblem}[5]
\begin{enumerate}
\item $E_1 = \{H\ on\ first\ toss\},\ p(E_1) = \cfrac{\abs{E_1}}{\abs{\Omega}} = \cfrac{4}{8} = \boxed{\cfrac{1}{2}}$
\item $E_2 = \{all\ outcomes\ the\ same\},\ p(E_2) = \cfrac{\abs{E_2}}{\abs{\Omega}} = \cfrac{2}{8} = \boxed{\cfrac{1}{4}}$
\item $E_3 = \{exactly\ one\ T\},\ p(E_3) = \cfrac{\abs{E_3}}{\abs{\Omega}} = \boxed{\cfrac{3}{8}}$
\end{enumerate}
\end{homeworkProblem}

\begin{homeworkProblem}[7]
Sample space $\Omega = \{1, 2, 3, 4, 5, 6\}^2 = \{(1, 1), (1, 2), (1, 3), \ldots, (6, 4), (6, 5), (6, 6)\}$. \\ \\
Event of interest $A = \{(1, 1), (2, 2), (3, 3), (4, 4), (5, 5), (6, 6)\}$. \\ \\
$\abs{\Omega} = 36,\ \abs{A} = 6$, so $p(A) = \cfrac{\abs{A}}{\abs{\Omega}} = \cfrac{6}{36} = \boxed{\cfrac{1}{6}}$
\end{homeworkProblem}

\begin{homeworkProblem}[9]
$\Omega = \{1, 2, 3, 4, 5, 6\}$, and 
$p(1) = p,\ p(2) = 2p,\ p(3) = 3p,\ p(4) = 4p,\ p(5) = 5p,\ p(6) = 6p. \\ \\
p + 2p + 3p + 4p + 5p + 6p = 1 \implies p = \cfrac{1}{21}$. \\ \\
$p(even\ number) = p(2) + p(4) + p(6) = 2p + 4p + 6p = \cfrac{2}{21} + \cfrac{4}{21} + \cfrac{6}{21} = \cfrac{12}{21} = \boxed{\cfrac{4}{7}}$
\end{homeworkProblem}

\begin{homeworkProblem}[11]
$\abs{\Omega} = 5! = 5 \times 4 \times 3 \times 2 \times 1 = 120$. \\ \\
Since all five people are of different height, there is only one correct increasing order of height, so $\abs{A} = 1$. \\ \\
$p(correct\ order) = \cfrac{\abs{A}}{\abs{\Omega}} = \boxed{\cfrac{1}{120}}$
\end{homeworkProblem}

\begin{homeworkProblem}[13]
Assuming order matters (i.e. the first four cards dealt must be Aces and the fifth card a King): \\ \\
$\abs{\Omega} = \Perm{52}{5} = \cfrac{52!}{(52 - 5)!} = \cfrac{52!}{47!} = 52 \times 51 \times 50 \times 49 \times 48$, \\ \\
$\abs{A} = \Perm{4}{4} \times \Perm{4}{1} = 4! \times 4 = 24 \times 4 = 96$, \\ \\
so $p = \cfrac{\abs{A}}{\abs{\Omega}} = \boxed{\cfrac{96}{\Perm{52}{5}}\ (if\ order\ matters)}$ \\ \\ \\
Assuming order does not matter: \\ \\
$\abs{\Omega} = \Comb{52}{5} = \cfrac{52!}{(52 - 5)!\ 5!} = \cfrac{52!}{47!\ 5!} = \cfrac{52 \times 51 \times 50 \times 49 \times 48}{5 \times 4 \times 3 \times 2 \times 1}$, \\ \\
$\abs{A} = \Comb{4}{4} \times \Comb{4}{1} = 1 \times 4 = 4$, \\ \\
so $p = \cfrac{\abs{A}}{\abs{\Omega}} = \boxed{\cfrac{4}{\Comb{52}{5}}\ (if\ order\ does\ not\ matter)}$\\

\end{homeworkProblem}

\begin{homeworkProblem}[15]
$\abs{\Omega} = \Perm{4}{4} = 4! = 24$, \\ \\
$\abs{A} = 1$, \\ \\
so $p = \cfrac{\abs{A}}{\abs{\Omega}} = \boxed{\cfrac{1}{24}}$
\end{homeworkProblem}

\newpage

\begin{homeworkProblem}[17]
$\Omega = \{any\ 3\ of\ the\ 7\ dwarves\}$, so $\abs{\Omega} = \Comb{7}{3} = \cfrac{7!}{(7 - 3)!\ 3!} = \cfrac{7!}{4!\ 3!} = \cfrac{7 \times 6 \times 5}{3 \times 2 \times 1} = \cfrac{210}{6} = 35$.
\begin{enumerate}

\item A = \{any 2 dwarves plus Dopey\}, i.e. choose Dopey and any 2 of the remaining 6 dwarves, \\ \\
so $\abs{A} = \Comb{6}{2} = \cfrac{6 \times 5}{2 \times 1} = \cfrac{30}{2} = 15$, \\ \\
and $p(A) = \cfrac{\abs{A}}{\abs{\Omega}} = \cfrac{15}{35} = \boxed{\cfrac{3}{7}}$

\item B = \{any 1 dwarf plus Dopey and Sneezy\}, i.e. choose Dopey and Sneezy and any 1 of the remaining 5 dwarves, \\ \\
so $\abs{B} = \Comb{5}{1} = 5$, \\ \\
and $p(B) = \cfrac{\abs{A}}{\abs{\Omega}} = \cfrac{5}{35} = \boxed{\cfrac{1}{7}}$

\item C = \{any 3 dwarves not including Dopey or Sneezy\}, i.e. choose 3 dwarves from the 5 remaining after removing Dopey and Sneezy, \\ \\
so $\abs{C} = \Comb{5}{3} = \cfrac{5!}{(5 - 3)!\ 3!} = \cfrac{5!}{2!\ 3!} = \cfrac{5 \times 4 \times 3}{3 \times 2 \times 1} = \cfrac{60}{6} = 10$, \\ \\
and $p(c) = \cfrac{\abs{A}}{\abs{\Omega}} = \cfrac{10}{35} = \boxed{\cfrac{2}{7}}$
\end{enumerate}
\end{homeworkProblem}

\end{document}