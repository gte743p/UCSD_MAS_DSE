\documentclass{article}

\usepackage{fancyhdr}
\usepackage{extramarks}
\usepackage{amsmath}
\usepackage{amsthm}
\usepackage{amsfonts}
\usepackage{tikz}
\usepackage[plain]{algorithm}
\usepackage{algpseudocode}
\usepackage{enumitem}
\usepackage{mathtools}
\usepackage{amssymb}

\usetikzlibrary{automata,positioning}

%
% Basic Document Settings
%

\topmargin=-0.45in
\evensidemargin=0in
\oddsidemargin=0in
\textwidth=6.5in
\textheight=9.0in
\headsep=0.25in

\linespread{1.1}

\pagestyle{fancy}
\lhead{\hmwkClass\ \hmwkTitle}
\rhead{\hmwkAuthorName}
\lfoot{\lastxmark}
\cfoot{\thepage}

\renewcommand\headrulewidth{0.4pt}
\renewcommand\footrulewidth{0.4pt}

\setlength\parindent{0pt}

%
% Added stuff
%

\DeclarePairedDelimiter\abs{\lvert}{\rvert}
\setenumerate[0]{label={(\alph*)}}

%
% Create Problem Sections
%

\newcommand{\enterProblemHeader}[1]{
    \nobreak\extramarks{}{Problem \arabic{#1} continued on next page\ldots}\nobreak{}
    \nobreak\extramarks{Problem \arabic{#1} (continued)}{Problem \arabic{#1} continued on next page\ldots}\nobreak{}
}

\newcommand{\exitProblemHeader}[1]{
    \nobreak\extramarks{Problem \arabic{#1} (continued)}{Problem \arabic{#1} continued on next page\ldots}\nobreak{}
    \stepcounter{#1}
    \nobreak\extramarks{Problem \arabic{#1}}{}\nobreak{}
}

\setcounter{secnumdepth}{0}
\newcounter{partCounter}
\newcounter{homeworkProblemCounter}
\setcounter{homeworkProblemCounter}{1}
\nobreak\extramarks{Problem \arabic{homeworkProblemCounter}}{}\nobreak{}

%
% Homework Problem Environment
%
% This environment takes an optional argument. When given, it will adjust the
% problem counter. This is useful for when the problems given for your
% assignment aren't sequential. See the last 3 problems of this template for an
% example.
%
\newenvironment{homeworkProblem}[1][-1]{
    \ifnum#1>0
        \setcounter{homeworkProblemCounter}{#1}
    \fi
    \section{Problem \arabic{homeworkProblemCounter}}
    \setcounter{partCounter}{1}
    \enterProblemHeader{homeworkProblemCounter}
}{
    \exitProblemHeader{homeworkProblemCounter}
}

%
% Homework Details
%   - Title
%   - Due date
%   - Class
%   - Section/Time
%   - Instructor
%   - Author
%

\newcommand{\hmwkTitle}{Worksheet\ \#4 - Random Variables, Expectation, and Variance}
\newcommand{\hmwkDueDate}{February 5, 2017}
\newcommand{\hmwkClass}{DSE 210}
\newcommand{\hmwkClassTime}{}
\newcommand{\hmwkClassInstructor}{Professor: A. Enis \c{C}etin}
\newcommand{\hmwkClassTA}{Teaching Assistant: Shivani Agrawal}
\newcommand{\hmwkAuthorName}{\textbf{Joshua Wilson} \and \textbf{A53228518}}

%
% Title Page
%

\title{
    \vspace{2in}
    \textmd{\textbf{\hmwkClass:\ \hmwkTitle}}\\
    \vspace{0.1in}\large{\textit{\hmwkClassInstructor}}\\
    \vspace{0.1in}\large{\textit{\hmwkClassTA}}
    \vspace{3in}
}

\author{\hmwkAuthorName}
\date{}

\renewcommand{\part}[1]{\textbf{\large Part \Alph{partCounter}}\stepcounter{partCounter}\\}

%
% Various Helper Commands
%

% Useful for algorithms
\newcommand{\alg}[1]{\textsc{\bfseries \footnotesize #1}}

% For derivatives
\newcommand{\deriv}[1]{\frac{\mathrm{d}}{\mathrm{d}x} (#1)}

% For partial derivatives
\newcommand{\pderiv}[2]{\frac{\partial}{\partial #1} (#2)}

% Integral dx
\newcommand{\dx}{\mathrm{d}x}

% Alias for the Solution section header
\newcommand{\solution}{\textbf{\large Solution}}

\newcommand*{\Perm}[2]{{}_{#1}\!P_{#2}}%
\newcommand*{\Comb}[2]{{}_{#1}C_{#2}}%

% Probability commands: Expectation, Variance, Covariance, Bias
\newcommand{\E}{\mathrm{E}}
\newcommand{\Var}{\mathrm{Var}}
\newcommand{\Cov}{\mathrm{Cov}}
\newcommand{\Bias}{\mathrm{Bias}}

\begin{document}

\maketitle

\pagebreak

\begin{homeworkProblem}[2]
Let X be the number of rolls until a 6 is seen. For any single roll, $p(6) = \cfrac{1}{6}$.  \\ \\
If p is the probability of the event of interest, then
$$\mathbb{E}[X] = \sum_{k=1}^{\infty} k \times p(X = k) = \sum_{k=1}^{\infty} k \times p(1 - p)^{k - 1} = \cfrac{1}{p}$$
Since in this case, $p = p(6) = \cfrac{1}{6}$, $\mathbb{E}[X] = \boxed{6}$
\end{homeworkProblem}

\begin{homeworkProblem}[4]
\begin{enumerate}
\item 
Let X be the number of people who get out on the $i^{th}$ floor.  \\ \\
There are 10 floors, so the probability that any one person gets out on the $i^{th}$ floor is $\cfrac{1}{10}$. \\ \\
There are n people, so the sample space is $\Omega = \{0, 1, 2, 3, \ldots , n\}$. \\ \\
We can model X as a $binomial(n, p)$ random variable, with $p = \cfrac{1}{10}$, so \\
$$p(X = 1) = \binom{n}{1}p(1 - p)^{n-1} = np(1-p)^{n-1} = \boxed{n\bigg(\cfrac{1}{10}\bigg)\bigg(\cfrac{9}{10}\bigg)^{n-1}}$$
\item
Define random variable $X_i$ such that
\begin{equation*}
  X_i = \begin{cases}
    1, & \text{if 1 person gets out on floor $i$}.\\
    0, & \text{otherwise}.
  \end{cases}
\end{equation*}
Then $\mathbb{E}[X_i] = n\bigg(\cfrac{1}{10}\bigg)\bigg(\cfrac{9}{10}\bigg)^{n-1}$, as shown in question 4(a). \\ \\
Let Y be the number of floors in which exactly 1 person gets out.
$$\mathbb{E}[Y] = \sum_{i=1}^{10} \mathbb{E}[X_i] = 10 \times n \bigg(\cfrac{1}{10}\bigg) \bigg(\cfrac{9}{10})^{n-1} = \boxed{n \cfrac{9}{10}^{n-1}}$$

\end{enumerate}
\end{homeworkProblem}

\newpage

\begin{homeworkProblem}[6]
Let $X_i$ be the event that the $i^{th}$ student ends up in the correct bed.  $p(X_i) = \cfrac{1}{n}$. \\

Define random variable $X_i$ such that
\begin{equation*}
  X_i = \begin{cases}
    1, & \text{if $i^{th}$ student gets in correct bed}.\\
    0, & \text{otherwise}.
  \end{cases}
\end{equation*}

$\mathbb{E}[X_i] = 1 \times \cfrac{1}{n} + 0 \times {\big( 1 - \cfrac{1}{n} \big)} = \cfrac{1}{n}$. \\

Now, let X be the number of students in the correct bed.  Then
$$X = \sum_{i=1}^{n} X_i \text{, and } \mathbb{E}[X] = \mathbb{E}[\sum_{i=1}^{n} X_i] = \sum_{i=1}^{n} \mathbb{E}[X_i] = \sum_{i=1}^{n} \cfrac{1}{n} = \boxed{1}$$

\end{homeworkProblem}

\begin{homeworkProblem}[8]
$Given: p(1) = p(2) = p(3) = p(4) = \cfrac{1}{8} \text{, and } p(5) = p(6) = \cfrac{1}{4}$.
\begin{enumerate}
\item
Let Z be the outcome of a die roll using a die with the above given probabilities.
\begin{align*} 
\mathbb{E}[Z] & = \sum_{k=1}^{8} k \times p(k) \\ 
& = 1 \times p(1) + 2 \times p(2) + 3 \times p(3) + 4 \times p(4) + 5 \times p(5) + 6 \times p(6) \\
& = 1 \times \cfrac{1}{8} + 2 \times \cfrac{1}{8} + 3 \times \cfrac{1}{8} + 4 \times \cfrac{1}{8} + 5 \times \cfrac{1}{4} + 6 \times \cfrac{1}{4} \\ 
& = \cfrac{1 + 2 + 3 + 4}{8} + \cfrac{5 + 6}{4} = \cfrac{32}{8} = \boxed{4} \\ \\
var[Z] & = \mathbb{E}[Z^2] - \mathbb{E}[Z]^2, \text{ and} \\
\mathbb{E}[Z^2] & = \cfrac{1^2 + 2^2 + 3^2 + 4^2}{8} + \cfrac{5^2 + 6^2}{4} = 19, \text{ and} \\
\mathbb{E}[Z]^2 & = 4^2 = 16 \text{, so} \\
var[Z] & = 19 - 16 = \boxed{3}
\end{align*}

\item
Let X be the sum of 10 die rolls.  Then X = 10 $\times$ Z, where Z is a single die roll as defined in 8(a), and \\ \\
$\mathbb{E}[X] = 10 \times \mathbb{E}[Z] = 10 \times 4 = \boxed{40}$. \\ \\
Because each die roll is independent, we can apply the variance rule for independent $X_i$'s: \\ \\
$var(X_1 + \ldots + X_k) = var(X_1) + \ldots + var(X_k)$, so \\ \\
$var[X] = var[10 \times Z] = 10 \times var[Z] = 10 \times 3 = \boxed{30}$

\item
Let A be the average of n die rolls. Then $A = \cfrac{1}{n} \sum_{i=1}^{n} Z_i$, where $Z_i$ is the outcome of the $i^{th}$ die roll modeled by random variable Z, defined in 8(a).
\begin{align*} 
\mathbb{E}[A] & = \mathbb{E}[\cfrac{1}{n} \sum_{i=1}^{n} Z_i] = \cfrac{1}{n} \sum_{i=1}^{n} \mathbb{E}[Z_i] = \cfrac{1}{n} \times n \times \mathbb{E}[Z_i] = \mathbb{E}[Z_i] = \boxed{4} \\
var[A] & = var[\cfrac{1}{n} \sum_{i=1}^{n} Z_i] = \Big(\cfrac{1}{n}\Big)^2 \times var[\sum_{i=1}^{n} Z_i] = \cfrac{1}{n^2} \times n \times var[Z_i] = \boxed{\cfrac{3}{n}}
\end{align*}
\end{enumerate}
\end{homeworkProblem}

\begin{homeworkProblem}[10]
\begin{enumerate}
\item
We can model $X_i$ as $binomial(m, \frac{1}{n})$, so $p(X_i = 0) = \binom{m}{0} \bigg(\cfrac{1}{n}\bigg)^0 \bigg(\cfrac{n - 1}{n}\bigg)^m = \boxed{\bigg(\cfrac{n - 1}{n}\bigg)^m}$
\item
$X_i \sim binomial(m, \frac{1}{n})$, so $p(X_i = 1) = \boxed{\binom{m}{1} \bigg(\cfrac{1}{n}\bigg)^1 \bigg(\cfrac{n-1}{n}\bigg)^{m-1}}$
\item
$X_i \sim binomial(m, \frac{1}{n})$, so $\mathbb{E}[X_i] = m \times \cfrac{1}{n} = \boxed{\cfrac{m}{n}}$
\item
$X_i \sim binomial(m, \frac{1}{n})$, so $var[X_i] = \bigg(m \times \cfrac{1}{n}\bigg) \bigg(1 - \cfrac{1}{n}\bigg) = \bigg(\cfrac{m}{n}\bigg) \bigg(\cfrac{n - 1}{n}\bigg) = \boxed{\cfrac{m\ (n - 1)}{n^2}}$
\end{enumerate}
\end{homeworkProblem}

\begin{homeworkProblem}[12]
Let X be the number of coin tosses required to see the same result twice in a row. \\ \\
Then $p(X = 1) = 0 \text{, and } p(X = k) = \big(\frac{1}{2}\big)^{k - 1} \text{ for } k > 1$.  Thus, \\
\begin{align*}
\mathbb{E}[X] & = 1 \times p(X = 1) + 2 \times p(X = 2) + 3 \times p(X = 3) + \ldots \\
& = 1 \times 0 + 2 \times \bigg(\cfrac{1}{2}\bigg)^{2-1} + 3 \times \bigg(\cfrac{1}{2}\bigg)^{3-1} + \ldots \\
& = 1 \times 0 + 2 \times \bigg(\cfrac{1}{2}\bigg)^1 + 3 \times \bigg(\cfrac{1}{2}\bigg)^2 + \ldots \\
& = 1 \times 0 + 2 \times \cfrac{1}{2} + 3 \times \cfrac{1}{4} + \ldots \\
& = \sum_{k=2}^{\infty} k \bigg(\cfrac{1}{2}\bigg)^{k-1} = \boxed{3}
\end{align*}

\end{homeworkProblem}

\end{document}




